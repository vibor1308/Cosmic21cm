\chapter{Foregrounds}

\begin{bf}
  \author{Emma Chapman and Vibor Jeli\'c}\\
  
Abstract\\
\end{bf}

This chapter discusses some important things


\section{What are foregrounds ?}
An overview of the foregrounds, related  physics, characteristics and current observational constrains.

\section{Foreground Mitigation}
Description of how smoothness of foregrounds present an opportunity. Discussion of challenges which have arisen
i.e. calibration in LOFAR causing foreground suppression, unexpected wedge leakage in MWA.

\subsection{Foreground Removal}

\subsubsection{Correlations and Polynomial Removal}
Description of early work look at correlations between slices i.e. Santos. Description of simply polynomial technique.

\subsubsection{Non-parametric Methods}
Description of early Wpsmoothing/FastICA/GMCA methods and then GPR currently in
use in LOFAR.

\subsection{Foreground Avoidance}
Description of EoR window and the concept evolution i.e. pitchfork
effect, wedge etc...

\subsection{Hybrid Methods}
Description of the more recent methods which attempt to use both
avoidance and removal and discussion on complentarity of the methods
and need for independent checks. Include foreground suppression as
done in power spectrum programs such as CHIPS.

\section{Polarisation leakage}
Polarised foregrounds, why we care and how to mitigate.

\bibliographystyle{plain}
\bibliography{Chapman_Jelic/References}


